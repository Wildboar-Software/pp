% -*- LaTeX -* Although it's slitex really

\documentstyle[blackandwhite,pagenumbers,small,oval,psfig,tgrind]{NRslides}

\raggedright

\begin{document}
\title	{PP - X.400 for the masses?}

\author{Julian P.~Onions\\
jpo@cs.nott.ac.uk\\[.5in]
X-Tel Services Ltd.\\
Nottingham University\\
Nottingham NG7 2RD\\
ENGLAND}

\date	{October 9, 1990}

\maketitle

\begin{bwslide}
\ctitle	{CONTENTS}
\begin{nrtc}
\item	What is PP?

\item	The PP Architecture

\item	The PP Queue

\item	Submission

\item	Scheduling and Control

\item	Channels

\item	Status and Availability
\end{nrtc}
\end{bwslide}

\begin{bwslide}
\ctitle{What is PP}
\begin{nrtc}
\item	Basically, A Mail transfer system (MTA in the jargon).

\item	Largely protocol independent, but deals with X.400
	and existing mail formats.

\item	Written by people at UCL and Nottingham

\item	Design aims include:-
	\begin{itemize}

	\item	High performance/throughput
	\item	Support for message conversion.
	\item	Support for multi-media
	\item	Clean interface for user interfaces.
	\end{itemize}

\end{nrtc}
\end{bwslide}

\begin{bwslide}
\ctitle{Why PP}
\begin{nrtc}
\item At the time there was no suitable alternative.

\item Most implementations were in the ``toy'' category.

\item Too expensive

\item Only did X.400 - no gatewaying, or else added as an
afterthought.

\item Worked to minimal profiles.
\end{nrtc}
\end{bwslide}

\begin{bwslide}
\ctitle{Some goals for PP}
\begin{nrtc}
\item	Scheduling of messages in an optimal way, such that many
thousands of messages a day can be processed.

\item	Robustness.

\item	Support for RFC-822, X.400(84) and X.400(88) and others.

\item	Reformatting between body part types in a general manner.

\item	Provision of management, authorisation and monitoring tools.

\item	Use of standardised services such as X.500.
\end{nrtc}
\end{bwslide}

\begin{bwslide}
\ctitle{The PP Architecture}
\begin{nrtc}
\item	PP is based on a system of cooperating processes.

\item	Each process has a specific task to achieve.

\item	Most of these processes are ``dumb''.

\item	The ``intelligence'' of the system is provided by two processes.

\end{nrtc}
\end{bwslide}

\begin{bwslide}
\ctitle{The PP Architecture}

\vskip .5in

\diagram[p]{figure2}

\end{bwslide}

\begin{bwslide}
\ctitle{The PP Queue}
\begin{nrtc}
\item	All messages are placed in a queue on disc.

\item	For each message there are two components
	\begin{itemize}
	\item	A Control file holding all the envelope information.
	\item	The Message structure holding the Message data.
	\end{itemize}
\item	The Control file is text encoded.

\item	The Message is either a single file, or a directory hierarchy.

\item	The Message may undergo a number of transformations whilst in
	the queue
\end{nrtc}
\end{bwslide}

\begin{bwslide}
\ctitle{Submission}
\begin{nrtc}
\item	The first intelligent process is {\em submit}.

\item	All messages entering the queue pass through this process.

\item	It validates the message in a number of ways:
	\begin{itemize}
	\item	Ensures the originator can be reached.
	\item	Ensures the destination is known.
	\item	Works out a conversion path from incoming format to outgoing.
	\item	Applies access controls if required.
	\end{itemize}
\end{nrtc}
\end{bwslide}

\begin{bwslide}
\ctitle{Scheduling and Controlling}
\begin{nrtc}
\item	The other ``intelligent'' process is the queue manager, {\em qmgr}.

\item	Keeps an in memory map of the queue.

\item	Drives all channel programs using network IPC.

\item	Makes all the scheduling decision about messages in the queue.

\item	Sorts the queue  in ``optimal'' ways.

\item	Allows connections to interrogate queue status
\end{nrtc}
\end{bwslide}

\begin{bwslide}
\ctitle{Queue Monitoring}

\vskip .5in

\psfig{file=console.ps,height=6in}

\end{bwslide}

\begin{bwslide}
\ctitle{Channels}
\begin{nrtc}
\item	Channels in PP process messages.

\item	May reformat a message by conversion e.g.
	\begin{itemize}
	\item	Mapping X.400 to RFC-822 message
	\item	Converting Telex to Ascii
	\end{itemize}
\item	May attempt delivery e.g.
	\begin{itemize}
	\item	Drive the SMTP or X.400 protocols
	\item	Deliver to users
	\end{itemize}
\item	May restructure the message
	\begin{itemize}
	\item	Convert a file into a directory hierarchy
	\item	Convert multiple body parts into one
	\end{itemize}
\end{nrtc}
\end{bwslide}

\begin{bwslide}
\ctitle{PP management}
\begin{nrtc}
\item The MTA console provides a good monitor of whats going on.
\item All address that are handled are logged in a special log which
an be processed to give statistics (and charging) information.
\item Authorisation can be controlled by user, host, content-type size
and route.
\end{nrtc}
\end{bwslide}

\begin{bwslide}
\ctitle{Current Status}
\begin{nrtc}
\item	PP-5.0 was released on September 21st, 1990.

\item	It is providing a real service at a number of universities and
	will shortly be providing the UK WEP service.

\item	PP is freely available for use and should be the backbone of
	JANET as it transitions to X.400 protocols.

\end{nrtc}
\end{bwslide}

\begin{bwslide}
\ctitle{Current Status}
\begin{nrtc}
\item PP-5.0 provides the following channels currently

	\begin{itemize}
	\item SMTP with DNS support
	\item X.400 (1984)
	\item UUCP mail
	\item JNT grey book mail
	\item Distribution list expansion and maintenance
	\item RFC822 local delivery and delivery time processing
	\item RFC822 and X.400 conversion
	\item X-windows based management console
	\item Message authorisation checking
	\item X.400(1988) P1 protocol (beta test)
	\item X.500 distribution lists (beta test)
	\end{itemize}

\end{nrtc}
\end{bwslide}

\begin{bwslide}
\ctitle{What is planned for the future}
\begin{nrtc}
\item	Fax gateway to convert outgoing messages to faxes and vice
versa.

\item	More use of the X.500 directory, for routing and directory
name lookup.

\item	P7 Message store with remote access

\item	Improvements in throughput and robustness

\item	Some work on System Five portability and a DECNET channel by
people outside the PP group. 
\end{nrtc}
\end{bwslide}

\begin{bwslide}
\ctitle{Availability and Support}
\begin{nrtc}
\item	Source code is free by FTP/FTAM etc.
\item	A distribution charge for a tape

\item	Available by FTP/FTAM from ucl, psi and eunet.

\item	Full support is also available if you require it from X-Tel.
\end{nrtc}
\end{bwslide}

\end{document}
