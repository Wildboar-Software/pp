\def\ppversion/{6.0}
\def\isodevrsn/{7.0}
\def\ppdate/{December 20, 1991}
\def\uktarfile/{\verb|pp/pp-6.tar|}
\def\tarfile/{\verb|pp-6.tar|}
\def\tarsize/{4Mb}
\def\compresssize/{3Mb}
\def\compressfile/{\verb|pp-6.tar.Z|}
\def\ukcompressfile/{\verb|<PP>pp-6.tar.Z|}
\def\ukpsfile/{\verb|pp-6-ps-a4.tar.Z|}
\def\psfile/{\verb|pp-6-ps-us.tar.Z|}
\def\pssize/{1.5MB}

\chapter {Overview}

\section {What is PP?}

PP is a message transfer agent (MTA), which supports a number of
message transfer protocols, including:

\begin {itemize}
\item X.400 (1984) P1 \cite{MHS}.
\item X.400 (1988) / ISO 10021 P1 \cite{CCITT.MHS}.
\item A number of RFC~822-based protocols \cite{RFC822}.
\end {itemize}

It is based on experience with previous MTAs, and has the following
aims:

\begin {itemize}
\item  Use for switching large volumes of messages in a service environment.
\item  Management features, suitable for complex sites.
\item Protocol conversion facilities, particularly for mapping between
RFC~822 and X.400 according to RFC~987 \cite{ARPA.MHS},
RFC~1148\cite{RFC1148} and RFC~1148bis.
\item  Body part format conversion.
\item Support for development of advanced user agents, particularly those
wishing to use X.400 and/or multimedia capability.
\end {itemize}

For a technical overview of PP, a number of papers are included in the
distribution, the paper by Kille is recommended
\cite{IFIP.PP}.   PP is implemented in \pgm{C}, and runs on a range of
UNIX and UNIX-like operating systems. 
The current release includes the following modules:

\begin{itemize}
\item   X.400 (1984) P1 protocol.
\item	X.400 (1988) P1 protocol.
\item  	Simple mail transfer protocol (SMTP), conformant to
  host requirements.
\item   JNT mail (grey book) Protocol.
\item   UUCP mail transfer.
\item	DECNET Mail-11 transfer
\item	Distribution list expansion and maintenance, using either a
  file based mechanism or an X.500 directory.
\item   RFC~822-based local delivery.
\item	Delivery time processing of messages.
\item	Conversion between X.400 and RFC~822 according to  the latest
  revision of RFC~1148, known as RFC~1148bis.
\item	Conversion support for reformatting body parts and headers.
\item	X-Window and line-based management console.
\item   Message Authorisation checking.

\item   Reformatting support for ``mail hub'' operation.
\item	X.500-based distribution list facility using the QUIPU 
	directory.
\item	FAX interworking
\end{itemize}




\section {The PP Documentation}

This manual refers to PP version \ppversion/ of \ppdate/.

The PP manual is divided into three volumes and this covering document,
which introduces PP and contains general
information relevant to all of the other volumes:

\begin {description}
\item[Volume 1:] Installation and Operations.  This describes all that is
needed to install PP and how to administer PP as part of an operational
service.
\item[Volume 2:] Programmer's Guide.  This describes programming interfaces
for submission of messages into PP, and for channels which deliver messages
locally or remotely.
\item[Volume 3:] User's Guide.  In general, PP should not be visible to the
end user.  There are some associated tools, which are needed, and are
described in this volume.
\end {description}

\section {A Brief History of PP}

PP began in 1985, when several of us wished to produce a system to support
X.400.  It was clear that MMDF (our favourite MTA at that time) was not going to be
extensible to do all that we wanted.  Steve Kille had looked at the EAN MTA at the
end of 1984.  Whilst this was an impressive development at the time, it did
not really meet our needs, although our initial plan was to use the OSI
layers from EAN.   

Two meetings were held in September 1985 to consider the possibilities of
future MTA development:

\begin {itemize}
\item At CWI, discussion took place between: Doug Kingston (CWI/BRL),
Piet Beertma (CWI), Miriam Amos (DEC/UCB), Kevin Dunlap (DEC/UCB),
Daniel Karrenberg (Uni Dortmund), and Steve Kille.

\item At UCL, discussion took place between: Phil Cockcroft (UCL), Doug
Kingston, Steve Kille, and Julian Onions.   
\end {itemize}

These meetings, and following messages, resulted in some of the
initial PP design.  As a result of the second meeting, Phil Cockcroft
and Julian Onions started to write large chunks of code, and Steve
Kille wrote some small chunks of code.

By April 1986, a system was starting to come together, which was being
called PP.  Marshall Rose was starting work on his \pgm{ASN.1} tools, which
were clearly going to be a part of PP!  In July, the mechanisms for
representing multimedia messages in the queue were designed and
evolved.

By October 1986, it was time to design the QMGR fully.  Steve Kille had been
at the infamous CCITT Directory meeting in Munich in February, and so
specified a QMGR design in the new ASDC (abstract service definition
conventions).  Phil Cockcroft left UCL at the end of 1986, which was a big
loss to the effort.

In February 1987, there was a basic system working, which talked SMTP
and X.400.  Filling in the holes took much longer than we expected!
The first priority was to replace the shell script QMGR.  Julian
Onions was then working on the ISODE ROS tools (ROSY and POSY), and so
it was decided to use ROS to implement the QMGR.  The QMGR and the
tools proceeded in parallel.  In 1988 Alina daCruz (UCL) rewrote the
access libraries, Pete Cowen (Nottingham) worked on a range of
formatting tools, and Bruce Wilford (UCL) redid all of the channel
binding and domain handling.  Mike Roe (UCL) started work on LISP code
to produce a first cut at the 1988 X.400 protocol.  Discussions with
Piete Brooks (Cambridge) led to the initial design of the JNT mail
channel.  In October 1988, Steve Kille presented a paper on PP at the
IFIP WG 6.5 conference, but there was still no PP ready for release.
By the end of 1988, there was a PP which had been radically changed
and looked much more like the final system.


PP received its first public airing at CEBIT (Hannover Fair) in March 1989,
where it carried ODIF over X.400, and converted into Diamond format by
use of private conversion tools.  It interworked with a number of
X.400(84) systems at this point.  This was as a part of the ESPIRIT
PODA project.


In 1989, the X MTA Console (Pete Cowen) and authorisation (John
Taylor) were added in, and the internal structures fully upgraded to
X.400(88).  PP 3.0 was cut in July 1989, and was used as the basis for
UK to full X.400 relaying.  This system was released outside UCL and
Nottingham on a very selective basis.

PP 4.0 was released at the very end of 1989 to 15 beta sites.  Three more
beta releases were made during 1990 to an increasing number of beta sites.
During this time, the upgrade to RFC~1148 was made, and initial
implementations of P1(1988) and directory-based distribution lists (Colin
Robbins) added.

PP 5.0, the first openly available release, was cut in September 1990.
As experience was obtained with PP 5.0, a bug-fix release known as PP
5.2 was released in February of 1991. This corrected a number of minor
problems with the 5.0 release. 

PP 6.0 was released on \ppdate/ 
\section {Etymology}

PP is not an acronym.  There is no truth in the rumour that PP stands
for ``Postman Pat,'' a famous British postman.

\section {Acknowledgements}

PP funding has come from a number of sources.  In particular the UK
Joint Network Team (JNT) has funded a significant part of the PP
development.  Jim Craigie of the JNT has been helpful and encouraging
in this.  The Alvey-sponsored COSMOS project funded a major part of
the initial development of PP.  The Alvey-sponsored Locator project
funded work on PP to provide hooks for security features.  Use of PP
as infrastructure in the ESPRIT PODA project has led to many
improvements in the current release.

Prof. Peter Kirstein of University College London (UCL) and Dr. Hugh
Smith of Nottingham University have been particularly tolerant of the
excessive resources consumed by this development.


The major work of coding PP (apart from the authors) has been done by
Phil Cockcroft, Alina da Cruz, and John Taylor of UCL and Pete Cowen
of Nottingham University.  Code has also been written by Adrian
Joseph, Mike Roe, Colin Robins, and Bruce Wilford of UCL.
Irene Hassel of UCL
helped to write documentation.  John Andrews of UCL has helped to
bring PP into service at UCL.

Piete Brooks of Cambridge Computer lab has been a faithful alpha test site,
and has contributed code to integrate UNIX-NIFTP and C-NRS.  He has made
many useful comments and contributed a number of bugfixes.  Andrew
Macpherson of STL has contributed a number of bugfixes and tools to help
Sendmail sites convert to PP.   

The ISODE from Marshall T. Rose of NYSERNet is an essential
component of PP.

The character set conversion code is based on that provided by
Keld Simonsen of the Danish UNIX User Group.

Useful comment and fixes have been provided by our beta test sites.
Particularly useful help has come from: Andrew Findlay of Brunel; Juha
Heinanen of FUNET; Simon Poole of EUnet; Ole Bj{\o}rn Hessen of University
of Oslo; George Michaelson of University of Queensland; and Peter Yee of NASA.

\chapter {Using PP}

\section {Conditions of Use}

PP is an openly available suite of software.  There are no
restrictions on its usage.  The authors accept no liability resulting
from use of PP.

\section {Comments and Discussion}

Comments and problems on PP should be sent electronically to the PP
support mailbox.  The RFC~822 form of this is:
\begin {quote}
\verb|pp-support@cs.ucl.ac.uk| 
\end {quote}
Bug reports (and fixes) to this address are welcome. These reports
will be dealt with on a best-effort basis.

The X.400 form is:

\begin {center}
\begin {tabbing}
Organisation \= UCL \kill
Surname \> PP-Support \\
Org Unit \> CS \\
Organisation \> UCL \\
PRMD \> UK.AC \\
ADMD \> \verb*| | \\
Country \> GB \\
\end {tabbing}
\end {center}

\pagebreak[4]

If this is not possible, comments may be sent to:

\begin {tabbing}
Postal Address: \= Steve Kille \\
\> Department of Computer Science \\
\> University College London \\
\> Gower Street \\
\> WC1E 6BT \\
\> UK \\ \\
Telephone: \> +44-1-380-7294 \\
\end {tabbing}


There is an electronic discussion list: 

\begin {quote}
\verb|pp-people@cs.ucl.ac.uk|
\end {quote}

All interested parties are encouraged to join this list by sending to:

\begin {quote}
\verb|pp-people-request@cs.ucl.ac.uk|
\end {quote}


\chapter {Obtaining PP}

The PP version \ppversion/ software may be obtained in the following
ways from these places:

\subsection*{DISTRIBUTION SITES}

\subsubsection*{NIFTP}
If you run NIFTP over the public X.25 or over Janet, and are
registered in the NRS at Salford, you can use NIFTP with username
``guest'' and your own name as password, to access UK.AC.UCL.CS to
retrieve the file \uktarfile/.  The
file \ukcompressfile/ is the \pgm{tar} image after being run through the
compress program (approx \compresssize/).


\subsubsection*{FTP}
If you can FTP to the Internet from outside Europe, then use anonymous
FTP to uu.psi.com [136.161.128.3] to retrieve the file \compressfile/ in
binary mode from the isode/ directory.  This file is the \pgm{tar} image
after being run through the compress program and is approximately
\compresssize/ in size.

\medskip
If you can FTP to the Internet from Europe, then use anonymous FTP to
archive.eu.net [192.16.202.1] to retrieve the file \compressfile/ in binary
mode from the network/isode/ directory. This file is the \pgm{tar} image
after being run through the compress program and is approximately
\compresssize/ in size.


\subsubsection*{EUROPE (tape and documentation)}

For mailings in EUROPE, send a cheque or bankers draft and a purchase
order
for 200 Pounds Sterling to:
\[\begin{tabular}{ll}
Postal address:&        Department of Computer Science\\
&                       Attn: Natalie May/Dawn Bailey\\
&                       University College London\\
&                       Gower Street\\
&                       London, WC1E 6BT\\
&                       UK
\end{tabular}\]
For information only:
\[\begin{tabular}{ll}
\ Telephone:&           +44 71--380--7214\\
\ Fax:&                 +44 71--387--1397\\
\ Telex:&               28722\\
\ Internet:&            \verb"natalie@cs.ucl.ac.uk"\\
&                       \verb"dawn@cs.ucl.ac.uk"
\end{tabular}\]
Specify one:
\begin{enumerate}
\item   1600bpi 1/2--inch tape, or

\item   Sun 1/4--inch cartridge tape.
\end{enumerate}
The tape will be written in \pgm{tar} format and returned with
a documentation set.
Do not send tapes or envelopes.
Documentation only is the same price.

\subsubsection*{EUROPE (tape only)}
Tapes without hardcopy documentation can be obtained via the software
distribution service of the European UNIX User group (EUUG).
The PP \ppversion/ distribution is called ``EUUGD22 - PP (\ppversion/)''.
\[\begin{tabular}{ll}
Postal address:&        EUUG Software Distributions\\
&                       c/o Frank Kuiper\\
&                       Centrum voor Wiskunde en Informatica\\
&                       Kruislaan 413\\
&                       1098 SJ  Amsterdam\\
&                       The Netherlands\\[0.1in]
For information only:&\\
\ Telephone:&           +31 20--5924121\\
        &               (or +31 20--5929333)\\
\ Telex:&               12571 mactr nl\\
\ Telefax:&             +31--20--5924199\\
\ Internet:&            \verb"euug-tapes@cwi.nl"
\end{tabular}\]
Specify one:
\begin{enumerate}
\item   1600bpi 1/2--inch tape: 130 Dutch guilders.

\item   800bpi 1/2--inch tape: 150 Dutch guilders.

\item   Sun 1/4--inch cartridge tape (QIC-24 format): 190 Dutch guilders.

\item   Sun 1/4--inch cartridge tape (QIC-11 format): 215 Dutch guilders.
\end{enumerate}
If you require DHL, this is possible and will be billed through.
Note that if you are not a member of EUUG,
then there is an additional handling fee of 300 Dutch guilders
(please enclose a copy of your membership or contribution payment form when
ordering).
Do not send money, cheques, tapes or envelopes;
you will be invoiced.


\subsubsection*{NORTH AMERICA}
For mailings in NORTH AMERICA,
send a check for 375 US dollars to:
\[\begin{tabular}{ll}
Postal address:&University of Pennsylvania\\
&               \small Department of Computer and Information Science\\
&               Moore School\\
&               Attn: David J. Farber (PP Distribution)\\
&               200 South 33rd Street\\
&               Philadelphia, PA 19104-6314\\
&               U.S.A.\\[0.1in]
Telephone:&     +1 215--898--8560
\end{tabular}\]
Specify one:
\begin{enumerate}
\item   1600bpi 1/2--inch tape, or

\item   Sun 1/4--inch cartridge tape.
\end{enumerate}
The tape will be written in \pgm{tar} format and returned with
a documentation set.
Do not send tapes or envelopes.
Documentation only is the same price.


\subsubsection*{AUSTRALIA and NEW ZEALAND}
For mailings in AUSTRALIA and NEW ZEALAND,

\begin{enumerate}
\item If you already have ISODE \isodevrsn/, send a cheque for 100
        dollars Australian to the following address. You will
        recieve only the PP manuals and a PP tape. {\bf Clearly specify
        on the cheque and order that you only want PP \ppversion/.}
\item	If you do not have ISODE \isodevrsn/, send a cheque for 350
        dollars Australian to the following address. You will
        recieve a tape containing both ISODE \isodevrsn/ and PP
	\ppversion/, together 
        with manuals for both ISODE \isodevrsn/ and PP \ppversion/.
\end{enumerate}

\[\begin{tabular}{ll}
Postal address:&        CSIRO DIT\\
&                       Attn: Andrew Waugh (ISODE Distribution)\\
&                       723 Swanston St\\
&                       Carlton, 3053\\
&                       Australia
\end{tabular}\]
For information only:
\[\begin{tabular}{ll}
\ Telephone:&           +61 3--282--2615\\
\ Fax:&                 +61 3--282--2600\\
\ Internet:&            \verb"ajw@mel.dit.csiro.au"
\end{tabular}\]
Specify one:
\begin{enumerate}
\item   1600/3200/6250bpi 1/2--inch tape, or

\item   Sun 1/4-inch cartridge tape in either QIC-11, QIC-24 or QIC-150 format.
\end{enumerate}
The tape will be written in \pgm{tar} format and returned with a documentation set.
Do not send tapes or envelopes.
Documentation only is the same price.

\subsubsection*{FTAM on Janet or PSS}
The source code is available by FTAM at the University College London over X.25
using Janet (DTE \verb"00000511160013"), PSS (DTE \verb"23421920030013") 
or IXI (DTE \verb"20433450420113") with
Transport Selector~\verb"259" (ASCII encoding).
Use the ``anon'' user-identity and retrieve the file \tarfile/.
The file \compressfile/ is the \pgm{tar} image after being
run through the compress program (\compresssize/).

\subsubsection*{FTAM on the Internet}

The source code is available by FTAM over the
Internet at host \verb"osi.nyser.net" \verb"[192.33.4.10]"
(TCP port~102 selects the OSI transport service)
with Transport Selector~\verb"259" (numeric encoding).
Use the ``anon'' user-identity, supply any password,
and retrieve \compressfile/ from the isode/ directory.
This file is the \pgm{tar} image after being run through the compress program
and is approximately \compresssize/ in size.

\subsubsection*{Online Postscript}

For distributions via FTAM,
the file service is provided by the FTAM implementation in ISODE~5.0 or later
(IS FTAM).

For distributions via either FTAM or FTP, there are additional files
available for retrieval, called \ukpsfile/  and \psfile/
which are compressed \pgm{tar} images (\pssize/) containing the entire
documentation set in PostScript format, 
for A4 and US paper respectively.   

\subsection*{SUPPORT}

A UK company has been set up to provide support for the ISODE, PP,
and associated packages: X-Tel Services Ltd.  This company provides an
update service, general assistance and site specific support.
This company is closely associated with the development of PP.

\[\begin{tabular}{ll}
Postal Address:&	X-Tel Services Ltd.\\
&		Nottingham University\\
&		Nottingham, NG7 2RD\\
&	        UK\\
Telephone:&	+44 602-412648\\
Fax:&		+44 602-790278\\
Internet:&	support@xtel.co.uk\\
\end{tabular}\]


\bibliography{bcustom,pp,networking}
\bibliographystyle{alpha}
