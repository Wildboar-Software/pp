% -*- LaTeX -* Although it's slitex really

\documentstyle[blackandwhite,pagenumbers,small,oval,psfig,tgrind]{NRslides}

\raggedright

\begin{document}
\title	{Electronic Mail -- X.400 and PP}

\author{Julian P.~Onions\\
jpo@cs.nott.ac.uk\\[.5in]
Communications Research Group\\
Computer Science Department\\
Nottingham University
}

\date	{September 7/8, 1989}

\maketitle

\begin{bwslide}
\ctitle	{CONTENTS}
\begin{nrtc}
\item	What is PP?

\item	An Introduction to X.400

\item	The X.400 Model

\item	The X.400 Message

\item	X.400 Addressing

\item	The PP Architecture

\item	The PP Queue

\item	Submission

\item	Scheduling and Control

\item	Channels

\item	Status and Availability
\end{nrtc}
\end{bwslide}

\begin{bwslide}
\ctitle{What is PP}
\begin{nrtc}
\item	Basically, A Mail transfer system (MTA in the jargon).

\item	Largely protocol independent, but deals with X.400
	and existing mail formats.

\item	Written by people at UCL and Nottingham

\item	Design aims include:-
	\begin{itemize}

	\item	High performance/throughput
	\item	Support for message conversion.
	\item	Support for multi-media
	\item	Clean interface for user interfaces.
	\end{itemize}

\end{nrtc}
\end{bwslide}

\begin{bwslide}
\ctitle{A (quick) Introduction to X.400}
\begin{nrtc}
\item	X.400 is a family of protocols that form an international
standard for electronic mail.

\item	Based on three layers
	\begin{itemize}
	\item	The User
	\item	The User Agent (UA)
	\item	The Message Transfer Agent (MTA)
	\end{itemize}
\item	Defined in terms of a language called ASN.1

\item	PP is an instance of an MTA
\end{nrtc}
\end{bwslide}

\begin{bwslide}
\ctitle{The X.400 Model}

\vskip .5in
\diagram[p]{figure1}
\end{bwslide}

\begin{bwslide}
\ctitle{The X.400 Message}
\begin{nrtc}
\item	The Message is divided into three parts:
	\begin{itemize}
	\item	The envelope;
	\item	the header; and
	\item	and a list of body parts.
	\end{itemize}

\item	The envelope (P1 protocol) is used by MTA's to work out how to
	deliver the message.

\item	The header (P2 protocol) is constructed by the user.

\item	The body parts can be in any of numerous formats.

\end{nrtc}
\end{bwslide}

\begin{bwslide}
\ctitle{X.400 Addressing}
\begin{nrtc}
\item	X.400 Addresses are binary lists of attribute values.

\item	They look pretty horrible

\item	X.400 is designed to interact with the X.500 directory service.

\item	This hides these address to a large extent.
\end{nrtc}
\end{bwslide}

\begin{bwslide}
\ctitle{The PP Architecture}
\begin{nrtc}
\item	PP is based on a system of cooperating processes.

\item	Each process has a specific task to achieve.

\item	Most of these processes are ``dumb''.

\item	The ``intelligence'' of the system is provided by two processes.

\end{nrtc}
\end{bwslide}

\begin{bwslide}
\ctitle{The PP Architecture}

\vskip .5in

\diagram[p]{figure2}

\end{bwslide}

\begin{bwslide}
\ctitle{The PP Queue}
\begin{nrtc}
\item	All messages are placed in a queue on disc.

\item	For each message there are two components
	\begin{itemize}
	\item	A Control file holding all the envelope information.
	\item	The Message structure holding the Message data.
	\end{itemize}
\item	The Control file is text encoded.

\item	The Message is either a single file, or a directory hierarchy.

\item	The Message may undergo a number of transformations whilst in
	the queue
\end{nrtc}
\end{bwslide}

\begin{bwslide}
\ctitle{Submission}
\begin{nrtc}
\item	The first intelligent process is {\em submit}.

\item	All messages entering the queue pass through this process.

\item	It validates the message in a number of ways:
	\begin{itemize}
	\item	Ensures the originator can be reached.
	\item	Ensures the destination is known.
	\item	Works out a conversion path from incoming format to outgoing.
	\item	Applies access controls if required.
	\end{itemize}
\end{nrtc}
\end{bwslide}

\begin{bwslide}
\ctitle{Scheduling and Controlling}
\begin{nrtc}
\item	The other ``intelligent'' process is the queue manager, {\em qmgr}.

\item	Keeps an in memory map of the queue.

\item	Drives all channel programs using network IPC.

\item	Makes all the scheduling decision about messages in the queue.

\item	Sorts the queue  in ``optimal'' ways.

\item	Allows connections to interrogate queue status
\end{nrtc}
\end{bwslide}

\begin{bwslide}
\ctitle{Queue Monitoring}

\vskip .5in

\psfig{file=console.ps,height=6in}

\end{bwslide}

\begin{bwslide}
\ctitle{Channels}
\begin{nrtc}
\item	Channels in PP process messages.

\item	May reformat a message by conversion e.g.
	\begin{itemize}
	\item	Mapping X.400 to RFC-822 message
	\item	Converting Telex to Ascii
	\end{itemize}
\item	May attempt delivery e.g.
	\begin{itemize}
	\item	Drive the SMTP or X.400 protocols
	\item	Deliver to users
	\end{itemize}
\item	May restructure the message
	\begin{itemize}
	\item	Convert a file into a directory hierarchy
	\item	Convert multiple body parts into one
	\end{itemize}
\end{nrtc}
\end{bwslide}

\begin{bwslide}
\ctitle{Status and Availability}
\begin{nrtc}

\item	Currently, PP is in alpha test at three sites

\item	Has transferred many thousands of messages

\item	Will be entering beta test to a slightly wider audience soon.

\item	Should be released at the end of this year.

\item	PP is freely available for use and should be the backbone of
	JANET as it transitions to X.400 protocols.
\end{nrtc}
\end{bwslide}

\end{document}
