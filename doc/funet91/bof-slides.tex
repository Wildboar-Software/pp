% -*- LaTeX -* Although it's slitex really

\documentstyle[blackandwhite,pagenumbers,small,oval,psfig,tgrind]{NRslides}

\raggedright

\begin{document}
\title	{ISODE/PP/QUIPU BOF}

\author{Julian P.~Onions\\
j.onions@xtel.co.uk\\[.5in]
X-Tel Services Ltd.\\
Nottingham University\\
Nottingham NG7 2RD\\
ENGLAND}

\date	{March 19th, 1991}

\maketitle

\begin{bwslide}
\ctitle{CONTENTS}
\begin{nrtc}
\item	Where are we now?
	\begin{itemize}
	\item	Current state of ISODE
	\item	Current state of QUIPU
	\item	Current state of PP
	\end{itemize}

\item	Whats coming up next?
	\begin{itemize}
	\item	What is happening to ISODE
	\item	Where is QUIPU going
	\item	what are the plans for PP
	\end{itemize}
\end{nrtc}
\end{bwslide}

\begin{bwslide}
\ctitle{Where are we now -- ISODE}
\begin{nrtc}
\item	6.0 is the official released version - released Jan 1990
\item	6.8 is an interim version. Released March 1991.
\item	7.0 is the next official release - when? (Anybodys guess)
\end{nrtc}
\end{bwslide}

\begin{bwslide}
\ctitle{Changes in ISODE up to 6.8}
\begin{nrtc}
\item	New ASN.1 compiler (PEPSY)
	\begin{itemize}
	\item	Smaller code.
	\item	faster?
	\item Harder to debug!
	\end{itemize}
\item	Many many bug fixes
\item	Some changes to smnp
\item	Has shared libraries under SUN OS 4.1
\item	DS can be used for bootstrapping now - DASED and IAED.
\item	Application lookup using DS.
\end{nrtc}
\end{bwslide}

\begin{bwslide}
\ctitle{Changes in Quipu up to 6.8}
\begin{nrtc}
\item	Changed to make use of pepsy (smaller binaries)
\item	Internal data structures now stored as an AVL tree.
\item	Indexing ability on sub-trees.
\item	Spot shadowing of entries -- including a DSA mastering its own entry.
\item	Attribute inheritance down the tree
\item	DSA relaying
\item	User friendly naming in the directory (UFN)
\item	T.61 string handling
\end{nrtc}
\end{bwslide}

\begin{bwslide}
\ctitle{Current state of PP}
\begin{nrtc}
\item	Current version is an interim 5.2 release.
\item	Reasonably stable
\item	Supports many protocols.
	\begin{itemize}
	\item	X.400 -- 1898 and 1988
	\item	SMTP with DNS
	\item	UUCP
	\item	Local delivery
	\item	Greybook
	\item	Reformatting and interworking between all the above
	\end{itemize}
\end{nrtc}
\end{bwslide}

\begin{bwslide}
\ctitle{ISODE -- things planned}
\begin{nrtc}
\item	Basically not very much.
\item	Profiling and speeding up high on the agenda
\item	TLI transport interface.
\item	Maybe look at other ASN.1 encodings.
\item	Bug fixing (as ever!)
\end{nrtc}
\end{bwslide}

\begin{bwslide}
\ctitle{Changes to QUIPU planned}
\begin{nrtc}
\item	Alignment to Internet DSP (as per emerging RFC)
\item	Management controls and Authorisation for operations.
\item	DIT counting
\item	DS for all lookups of ISODE applications
\item	Strong authentication of operations 
\end{nrtc}
\end{bwslide}

\begin{bwslide}
\ctitle{Changes to PP planned}
\begin{nrtc}
\item	For 6.0 release
	\begin{itemize}
	\item	FAX gateway -- subject to the whims of manufacturers
	\item	More robust, faster etc.
	\item	Better support of deliver reports
	\item	Channel pairing.
	\end{itemize}
\item	For 7.0 release
	\begin{itemize}
	\item	Integration with the directory for routing and name
lookup.
	\item	Most tables then optional.
	\item	A P7/P7+/P3 message store available
	\item	An X based X.400 user agent available with the release.
	\end{itemize}
\end{nrtc}
\end{bwslide}
\end{document}
