% -*- LaTeX -*-

\input lcustom

\documentstyle[11pt,a4]{article}

\begin{document}

\title{PP -- The Fun and the Pain}
\author{Julian P.~Onions\\[0.1in]
X-Tel Service Ltd.\\
Nottingham University\\[0.1in]
jpo@xtel.co.uk}
\date{Tuesday 19th March 1991}

\maketitle

\begin{abstract}
This paper describes the design and implementation of
the PP message transport system. This is a message transfer system
that can relay a variety of protocols, and gateway between them.

The paper describes what PP is now and some of its plans for
development. The second half of the paper is a report on some of the
experiences gained watching and tinkering with PP.
\end{abstract}

\section{Introduction}

This paper describes the design and implementation of PP\footnote{PP
is not an acronym. Despite many rumours to the contrary, the author is
sticking to this position!.}. PP is a generic Message Transfer Agent
capable of carrying a number of message formats. It has been written
as a joint project by University College London and Nottingham
University and is designed to run on \unix/ systems. It makes
use of the ISODE package\cite{ISODE}.  Its design is heavily
influenced however, by the need to carry X.400 MHS\cite{CCITT.MHS} and
RFC-822\cite{RFC822} based mail. The major goals of PP include:
\begin{itemize}
\item	Robustness. PP is designed for heavy use. It should be
capable of transferring high levels of traffic in an optimal way.

\item	A clean interface for submission and delivery. PP does not
provide a user agent but does provide a library interface which allows
user agents to be added to PP easily.

\item	Sophisticated message processing to optimise delivery and
remote transfers.

\item	Support for message conversion in an integrated way.

\item	Support for multiple addressing formats (initially two).

\item	Support for multi-media mail.

\item	Support for arbitrary message structures.

\item	Designed for large systems. PP is not designed to run on your
PC or even on a workstation. It is designed to be a central mail
processing hub for a group of computers. It is not a small piece of
software!
\end{itemize}

As indicated above, PP is designed to deal with a large variety of
problems. The questions that comes to mind is ``why bother spending
all those years doing this?''. 

The answer is partly historical, but is still mostly relevant. When PP
was first thought about, there was very little that could be picked up
``off the shelf'' to deal with X.400 mail. What's more, the products we
had seen were just not suitable for the sort of environment we wanted
to run a mail system in. Most of the products then, and still now to
an extent, were what might be considered ``toy'' systems. Yes, they
would take an X.400 message and deliver it to a recipient, but there
is more to a reliable MTA than this. PP put management and monitoring
of the MTA very high on its list of desirable features. We believe
this has been attained - although not perfectly yet (what is).

\section{The design of PP}

The design of PP was influenced in part by the RFC-822 mail system
MMDF\cite{MMDFII}. In this model, extensive use is made of separate
processes to do different jobs. This allows each element to be built
and tested separately and encourages modularity. A fuller description
of the design of PP can be found in \cite{PP.MTA}.

\subsection{The Process Structure}

PP has a single queue of messages on which a number of processes
operate.  There are two critical processes which undertake complex
functions, the rest of the processes are essentially ``dumb''.

\subsubsection{Submit}
The first key process is that entitled {\em submit}. This process
guards the entrance to the queue. All messages entering the queue pass
through this process. It performs a variety of checks on the message
to ensure it can either be delivered or returned; works out the
routing for the message; calculates the reformatting of the message if
necessary; and applies other criteria such as authorisation and
authentication. By checking as much as possible at this point later
processes have to do less and completely undeliverable messages can be
stopped from entering the queue. 

As all message must pass though submit, all messages are checked at
this stage. This is where the authorisation policies are enforced.
Where possible, submit will always try and accept a message and return
it if necessary. Typically this leads to better error messages as
submit can take complete charge of generating the delivery report. 

Submit will refuse to accept a message when it can't parse or route
back to the originator. In this case, submit attempts to signal back
to the remote site that this message will not be accepted. In some
protocols, such as X.400(84), this is not possible. In this case,
submit gives up and the message as a whole is forwarded to the
postmaster to sort out. Fortunately, such occurrences are fairly rare.

\subsubsection{Qmgr}
The second key process is the {\em \verb|qmgr|}, or queue manager. This
process is responsible for all the scheduling operations of the
messages in the queue. The \verb|qmgr| holds a representation of the queue in
memory and so can perform complex and optimal scheduling of messages.
This scheduling consists of either delivering the message locally or
remotely or else converting the message within the queue (e.g.
gatewaying between protocols).

As the \verb|qmgr| holds a complete representation of the queue in memory it
also provides an information service. It can be queried by suitable
tools to display the state of the queue and such tools can even
control the behaviour of the \verb|qmgr|. This is the second half of the
management framework. As the \verb|qmgr| has a full overall picture of what
is going on in the queue - on a second to second basis, good global
decisions are possible.

An X-windows based MTA console is provided with PP which can attach
to remote \verb|qmgr|'s and display the status in graphic detail, where
possible using colour to highlight problems. With suitable
authentication, the MTA console can also control the \verb|qmgr|'s behaviour.

\subsection{The Queue}

The queue of messages is held in two components. The first is an
address file which contains the envelope information about the
message. This is held in a text encoded form and is a generic envelope
structure which can hold both X.400 and RFC-822 messages. The text
encoding is a big advantage as it allows manipulation of the queues in
an emergency by a text editor. Whilst this should not normally be
necessary it is much easier than providing special management tools
to cover every eventuality.

The second component is the message content. This includes the headers
and body parts. If no processing is necessary for the message this is
kept as a single file. However, if conversion is necessary, the
content is broken up into parts, each part being placed in a separate
file. Each time a message content is transformed, a new directory is
created and the appropriate files are changed. This allows simple
filters (such as {\em sed} even) to do the conversions.


\subsection{Channels}

Most other processes in PP are termed {\em channels}. A channel takes
as input a message and outputs a different form of that message.
Channels are controlled directly by the \verb|qmgr| by means of a networked
protocol. Each channel does a specific job. Channels come in four
basic forms as described below.

\subsubsection{Outbound Channels}
This style of channel is usually a protocol engine. It takes a message
out of the queue and delivers it. This delivery may be local as in the
case of delivering to a mailbox; or remote such as driving the
SMTP\cite{SMTP} or X.400 protocols. Each channel reads the internal
format of the queued message and converts it to the protocol specific
format. The \verb|qmgr| informs a channel which message to deliver and the
channel reports on the status of the message. This allows the \verb|qmgr| to
discover whether the delivery attempt was successful, the remote site
did not respond or whether the message cannot be delivered. This
information is then used by the \verb|qmgr| to schedule further deliveries.

\subsubsection{Inbound Channels}
Inbound channels perform the opposite function to outbound channels.
They take care of the various protocols necessary to receive a
message, either from a user agent (local submission) or across a
network. Their main task is mapping from the protocol specific form of
the message to the form used withing the queue. This form is then
passed by a private protocol to submit which enqueue the message and
then informs the \verb|qmgr| of the new message.

\subsubsection{Formatting and protocol conversion channels}
These channels transform the message in some way without delivering
it. They are treated in exactly the same way as outbound channels and
are scheduled by the \verb|qmgr|. These channels come in two basic types.
\begin{itemize}
\item	Channels that change the directory structure. These either map
from the single file into a directory hierarchy, or the reverse.

\item	Simple reformatting. These channels might change, say, a teletex
body part into an ascii body part. When gatewaying is being performed,
such a channel might convert RFC-822 headers into X.400 headers.
\end{itemize}

\subsubsection{Housekeeping and maintenance channels}
These channels do not usually have anything to do with delivery, but
perform meta tasks. These include loading the \verb|qmgr|, deleting
messages, timeing out messages and removing junk from the queue.

\section{PP Status}

PP currently contains the following parts.

\begin{description}
\item[Core]	The basic core of the system including the \verb|qmgr|,
submit and a number of support tools.

\item[X.400 1984] The X.400 1984 protocols are supported.

\item[SMTP]	Smtp protocols are supported, including use of the
name server protocols for address resolution.

\item[Grey Book] Grey book mail transfer is working, using  the
unix-niftp blue book transfer system to provide non-spooled mail
transfer over JANET.

\item[List]	A list channel to offload the processing of
distribution lists. This helps speed up submission to large
distribution lists and allows list management.


\item[MTA Console]	An X-windows based console display which
interrogates the \verb|qmgr| and displays the current state of the
queues. It emphasises problems in the queue and allows parameters to
be adjusted interactively.

\item[UUCP] There is some support for UUCP though this has not really
been tested in a large UUCP environment.

\item[Tools]	Various auxiliary programs to configure the system and
provide debugging support.
 
\end{description}

PP also contains the following experimental modules.
\begin{description}
\item[X.400 1988]	The 1988 version of X.400 is in a beta test
form. The 1988 version of X.400 is uncommon at present so interworking
testing is limited. This will include secure messaging and extensions.

\item[Directory Lists] An experimental channel that store distribution
lists in the X.500 directory is provided. This will be emphasised more
in future releases.

\end{description}

\subsection{The future}
Work is already under way on the next version of PP. The next version
has three main thrusts.

\subsubsection {Reliability, robustness and efficiency}
These are general goals, without any specific actions to be taken. In
general we hope the level of reliability and robustness of the
software to increase with each future release. At the same time, some
effort will be expended on streamlining the code and improving
throughput and decreasing machine load (the usual motherhood and apple
pie goals...).

\subsubsection {Use of directory}
The next version of PP will make much larger use of the X.500
directory. It will use the distribution list channel indicated above.
it will also be able to map Directory Names into O/R names by use of
the directory. Finally, work will be done on use of the directory for
routing. 

\subsubsection{Message Store}
A Message Store for PP is currently being written. It will allow
both P7 and P7+ access to messages. This will allow workstations and
portable computers to retrieve and store messages remotely. A user
agent for Unix with these features in mind is being written as a
separate project.

\section{Message Scheduling}
Much of the PP system revolves around the \verb|qmgr| process which as
described above schedules the messages in the queue. The development
of this aspect of PP has been through many iterations, and is quite
interesting (at least to the author).

\subsection{The first cut}
The first version of the \verb|qmgr| was a shell script!
It was 152 lines long by the time it was dropped, and originally had a
simplicity and understandability that was rather elegant (if I do say
so myself). All in all it was cute, and did the job reasonably well
for a week or two. 

At this stage though PP was very different to its current form
and no serious message process was being done. This version was only
used for experimentation, and I certainly would never have trusted my
mail with it. This version was never intended as anything other than
the proverbial {\em quick hack} to get things going. To my rather
faded memory it scheduled less than 100 messages in its short
lifetime.

It was obvious though that this method was of limited potential. It
did not have the capability for interrogation. It did hold an in-core
copy of the queue as a shell variable but it refreshed this copy every
few minutes (can you say \verb|echo msg.*|). It was also clear that
with really big queues the lack of management and sophisticated
backoff and retry strategies alone would just not cut it.
Besides, who would have any credibility for a 15Mb mail system
claiming performance upwards of a message per second controlled by 100
lines of shell!

\subsection{Take 2}
The next version of the \verb|qmgr| was a C program. This was a small
program, I forget exactly the size. It did queue management in a
rather cavalier fashion. It scheduled messages by the simply
fork/exec'ing  processes to deal with delivery. It read the queue
directly on start up and received UDP packets from submit to notify it
that new messages had arrived. The UDP packet contained a simple
string which was the queue name of the message. The \verb|qmgr| would then go
and read in this message, add it to the internal queues and start
scheduling it. 

It was cavalier in the respect that it did not trust anyone. The UDP
packet was a hint that a message had arrived. The \verb|qmgr| would
independently check this by attempting to read from the queue.  Also,
as UDP is inherently unreliable - and also because the \verb|qmgr| had an
{\em attitude problem} - it would reread the entire queue frequently
to check what it thought was correct and that it hadn't missed
anything.

This scheme works ok after a fashion. However, the frequent queue
reading we knew by experience would be a real performance killer. Most
of the queueing systems that we were familiar with have had as a
bottleneck the reading of the queue. UUCP has long had this problem,
with some very ingenious work-arounds to avoid quadratic behaviour with
large queues. MMDF had a similar problem, although to a lesser extent.
Normally MMDF will try and sort the mail queue optimally, but above a
threshold (300 messages) it admits defeat and gets very relaxed about
the whole thing, taking each message as it finds it without any
attempt at optimality.

Another problem was this UDP issue. I liked it as it was simple, but
if we wanted to get away from the frequent queue reading, this
process would have to be reliable. We could obviously move to TCP for
reliability and then we could pass all the information we needed to
the \verb|qmgr|. But how to pass the data? We could have used something like
XDR and RPC from sun, which would probably do the job very well.
However, at the time I was deeply involved in work with ISODE and
getting ASN.1 tools together. We therefore decided - hey, we're
suppose to be pushing OSI, lets see if it can cut it. This leads on to ...

\subsection{The current version}
The book {\em The Mythical Man Month} makes a great play of the fact
that the third and presumably subsequent versions of a piece of
software are usually the better versions. The first version being a
rather cautious attempt to see if you can do the job, having minimal
functionality and being a sort of prrof of concept case. The second
version is then usually a wild extravaganza of crazy ideas that you
thought up while writing the first version and just {\em had} to put
in. The third version then falls somewhere between these two. I am
rather proud therefore that there have been three rewrites of this
part of the system.

The current version of the \verb|qmgr| is rather detached from reallity in
some respects. It never interacts directly with anything. Indeed the
\verb|qmgr| can run on a different machine to the rest of the system.
The \verb|qmgr| issues commands to other programs for all its
operations. In this way it is usually very responsive as it never has
to do serious amounts of work itself. When started up, the \verb|qmgr|
first schedules a channel called \verb|qmgr-load| which reads the
queue and passes its findings to the \verb|qmgr|. The \verb|qmgr| then
processes the messages in the queue calling different channels until
finally on each message the \verb|msg-clean| process is called to
remove each message.  Every 6 hours or so, the \verb|qmgr| will
reschedule \verb|qmgr-load| to just check the queue is what it really
thinks it is. It also schedules irregularly a process called
\verb|trash| which will crawl around the queue looking for dead
messages or general junk.

The \verb|qmgr| receives news of new messages via ROS operations from submit
and believes this information. From this it can be seen that the \verb|qmgr|
is not a trusted process. It just schedules the channels it thinks it
should on the basis of the information it receives. Each channel is
responsible for checking that the \verb|qmgr|'s request is reasonable.
This scheme appears to work fairly well. 

\subsection{Tuning performance - initial}
It turns out that scheduling is never as easy as you thought it might
be - even if you take this maxim into account! To begin with, I was
running the \verb|qmgr| on an empty queue and introducing one message at a
time to see that it scheduled things as it should. It took many
iterations before I had everything sorted out so that the \verb|qmgr| always
scheduled things that were ready to run. There is a fairly complex
interaction between address that have been cached for a while,
messages that are locked as they are being processed and the
intelligence of scheduling. Eventually this was solved, such that
messages appeared to do what they were suppose to.

The next step was to introduce it into service. We ran it on a machine
at Nottingham processing maybe 100 messages a day delivering to a
handful of brave recipients. It turns out that 100 messages a day is
not too dissimilar to the one message at a time scenario. Things seem
to run fine. Then one day a glut of messages arrived at the same time
as several local messages were submitted. The machine (a sun 3/60 with
8Mb) practically died. The \verb|qmgr| was a little over eager -
scheduling everything it could We had 10-15 processes hammering away
at the queue and performance was non-existent.

The first step to correct this was to put a limit on the total number
of simultaneous channels that could be run. By setting this figure to
one you can get back to linear behaviour and a figure of about 3 is
suitable for most reasonably small sites. The next thing that was
noticed was that with 2 messages were introduced into the queue, the
following happened.
\begin{quote}\small\begin{verbatim}
process1: deliver message1 to recipient
process1: message1 delivered
process2: start up and initialise
process1: deliver message2 to recipient
process2: delete message1
process2: shutdown
process1: message2 delivered
process2: start up an initialise
process2: delete message1
process2: shutdown
\end{verbatim}\end{quote}
As the delivery time is slightly longer than the time to delete a
message, the deletion process is started up and stopped many times. A
simple fix for this is to cache a channel for a few seconds when it
shuts down. By enforcing an idle period of, say, 20 seconds, between
the same process running, this cuts down thrashing of one process.
Another useful technique is to round up cache times. If you cache a
message, then you round that time off to the nearest whole minute.
This means when the channel restarts to retry these messages is may
have a number to try - rather than starting up for just one message
and again a few seconds later for another.

\subsection{Tuning performance - for real}
The next step in the evolution was when PP was pressed into service at
UCL. Here there are many messages processed (about 30,000+ a week).
As far as I recall, this really didn't affect scheduling policy that
much. Most of the bugs had been shaken out and although the queues had
grown to perhaps 1-300 messages things still worked reasonably well.
What it did show up was that more management control was needed. Much
work was done with the \verb|MTAconsole| at this stage to allow a
variety of new controls and some graphical details on performance.

It was pushed a little more heavily one week when most of UCL's
machines were down for air-conditioning maintenance. At this point
delivery to local users could not proceed as the relevant NFS
partitions were not present. It would have been nice to do some more
useful cacheing in this case, such as detecting that a whole disc was
missing so disabling delivery to a whole range of users. Unfortunately
this sort of thing is rather difficult to detect in software -
especially in a portable way. All you generally know is that a
chdir(2) or open(2)failed.

Anyway, the \verb|qmgr| continued to run ok with a few minor tweaks here and
there - mainly to provide new statistics for the console process.

And then... it was brought up at nsfnet-relay. 

Whereas at UCL the queues might peak at 1000 messages occasionally,
this sort of figure is the typical ``quiet'' day at nsfnet-relay. We
were rather staggered seeing at first hand the sort of traffic they
processed. The ULCC people had received a copy of the UCL system and
had been playing around with it to see if it did what was required.
Then one particularly heavy day, there was just too much traffic for
the MMDF and grey-book system to take. As some traffic had been
filtered through the PP system at nsfnet-relay already, it was decided
to switch to PP to clear the backlog. This was done, and a backlog of
5000 messages was cleared together with a constant stream of new
traffic within about a day. As everything appeared to be working ok
the system was left in production.

This worked fine, although watching things carefully it was clear that
things were not being processed optimally. Of the 6 channels the \verb|qmgr|
had to work with it would start off by allocating the following.
\begin{quote}
\small
\begin{verbatim}
smtp: 2
greybook: 2
msg-clean: 1
822touk: 1
\end{verbatim}
\end{quote}
However, after a while, some of the channels cleared - or else all
hosts were deemed unreachable currently. The channel would then
shutdown and be reallocated. Then, at this point they would do other
work such as reformatting and messages would build up in one of the
outbound queues. Once the reformatting was complete, Channels would be
allocated to the outbound channel. As there were many messages on this
channel, as other channels finished they would be allocated to this
channel. This would result in an allocation such as:
\begin{quote}
\small
\begin{verbatim}
greybook: 5
msg-clean: 1
\end{verbatim}
\end{quote}
Eventually the greybook queue would clear, but by this time the smtp
queue had built up so processing switched to that.
This continued in a sort of bistable flip-flop routine. This is not
particularly serious - although probably not the sort of behaviour that
is required.

The real disaster occurred a week or so later. It seems to have
happened after the US thanksgiving holiday. We suspect that during
this time a number of machines in the U.S. were either taken down or
crashed, making the mail pile up at nsfnet-relay. Then, when they came
back up - there was an enormous flood of mail. At one stage we had
over 11,000 messages in the queue. A simple ls(1) in the queue
directory took several minutes! PP continued to work through this
queue though, and was making some progress. Every night however, the
machine is taken down for 30 minutes or so to perform a full backup.
When the machine restarted, the \verb|qmgr| scheduled the
\verb|qmgr-load| process as normal.  However, there were so many
messages in the directory (and unix behaves rather quadratically with
large directories) that this process took over an hour to read in all
the files. By this time the process had grown to about 50Mb in size.
Then it tried to transfer the queue to the
\verb|qmgr|, which also had to grow to 50Mb's to accomodate it. There was only
100Mb of VM configured - so not suprisingly the \verb|qmgr| ran out of
memory. Despite this it continued to run which I was quite pleased
about!

This was desperate. The nsfnet-relay machine is not one that you can
say ``Oh well, never mind - thats one for the record books'', it {\em
has} to be fixed. There was a panic days work to convert the
qmgr-loader into an incremental process that loaded 50 messages at a
time rather than all at once. Meanwhile, a small debugging program we
had around that was used to load individual messages into the queue
was pressed into service. Together with ls and xargs it succeeded in
loading up the
\verb|qmgr| in about 2 hours total. The backlog of messages was cleared in
about two days total.

In hindsight, this was an obvious problem waiting to happen, but it
was an issue that was hardly discussed until it happened.

\subsection{... in the light of experience}
After the experiences with nsfnet-relay, we are adapting some of the
software. It is clear that more statistics are needed (such as number
of messages processed and messages processed/second). It also seems as
though the \verb|qmgr| is using too much of its memory when running. From
observation, it uses less than 1Mb of memory to hold just over 1000
messages (e.g. the \verb|qmgr| at the peak traffic on nsfnet-relay was of the
order 11Mb). However, it also seems to use most of this memory most of
the time. This causes heavy paging traffic. With some subtle changes
to the internal data structures I believe it is possible to reduce the
working set size considerably - which will give other processes a
chance!

\section{Conclusions}
PP has been an interesting project and we are now getting some real
feedback on the design decisions taken early on. It seems to be mostly
up to the job asked of it - even in the busiest environments. We
worked out using back-of-the-envelope calculations, that it can
process about a message/second at the present time - on input and
output. We think with a few of the changes hinted at above we can move
this to 10 messages/second without too much effort. This works out at
just less than a million messages a day. We hope this will be
sufficient for most peoples usage. (You could always buy another
machine!)

\bibliography{bcustom,networking,software}
\bibliographystyle{alpha}

\showsummary

\end{document}
